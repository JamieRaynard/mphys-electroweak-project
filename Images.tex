\documentclass{article}
\usepackage{graphicx}
\usepackage{amsmath}
\usepackage{amssymb}
\usepackage{float}
\newcommand{\ZMass}{91.142}
\newcommand{\ZWidth}{2.001}
\newcommand{\ZMasssysterror}{0.003}
\newcommand{\ZWidthsysterror}{0.001}
\newcommand{\Zmassstaterror}{0.009}
\newcommand{\Zwidthstaterror}{0.025}
\newcommand{\ZCorrelation}{0.065}

\title{Images}
\author{}
\date{\today}

\begin{document}
\maketitle

The mass of the Z boson was found to be \ZMass~$\pm$~\Zmassstaterror~$\pm$~\ZMasssysterror~GeV where the first error is statistical and the second systematic\\
The width of the Z boson was found to be \ZWidth~$\pm$~\Zwidthstaterror~$\pm$~\ZWidthsysterror~GeV where the first error is statistical and the second systematic\\
The correlation coefficient was \ZCorrelation
\begin{tabular}{l c c}
\hline
 & Mass & Width \\
\hline
Calibration error & -0.003229 & 0.000379 \\
Smearing error & -0.000075 & 0.001182 \\
\hline
\end{tabular}


\begin{table}[H]
\centering
\begin{tabular}{l c c}
\hline
 & Mass & Width \\
\hline
Values & 91.14187 & 2.00069 \\
Statistical error & 0.00876 & 0.02512 \\
\hline
\end{tabular}
\caption{Z Mass and width results with statistical error}
\end{table}

\begin{figure}[H]
    \centering
    \includegraphics[width=0.49\linewidth]{transient/Upsilon_mass_DATA.png}
    \includegraphics[width=0.49\linewidth]{transient/Upsilon_mass_U1S.png}
    \caption{}
    \label{fig:label}
\end{figure}

\begin{figure}[H]
    \centering
    \includegraphics[width=0.75\linewidth]{transient/Upsilon_mass_comparisson.png}
    \caption{}
    \label{fig:label}
\end{figure}

\begin{table}[H]
\centering
\begin{tabular}{cc}
\hline
$\chi^2$ & ndf \\
\hline
405.71 & 48 \\
\hline
\end{tabular}
\caption{minimum $\chi^2$ and number of degrees of freedom}
\end{table}


\begin{figure}[H]
    \centering
    \includegraphics[width=0.75\linewidth]{transient/Z-stack_similtaneous (real).pdf}
    \caption{}
    \label{fig:label}
\end{figure}



\begin{figure}[H]
    \centering
    \includegraphics[width=0.75\linewidth]{transient/Z-Error-Ellipse (real).pdf}
    \caption{}
    \label{fig:label}
\end{figure}

When seperating the positve and negative dipole data:
\input{dipole_table.tex}

\begin{figure}[H]
    \centering
    \includegraphics[width=0.75\linewidth]{transient/Z-stack_similtaneous (pos-magnet).pdf}
    \caption{Positive dipole}
    \label{fig:label}
\end{figure}

\begin{figure}[H]
    \centering
    \includegraphics[width=0.75\linewidth]{transient/Z-stack_similtaneous (neg-magnet).pdf}
    \caption{negative dipole}
    \label{fig:label}
\end{figure}

\end{document}